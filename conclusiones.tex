\chapter{Conclusiones}

\section{Observaciones}

Realizar este proyecto ha permitido a la empresa disponer de un sistema de seguimiento y monitorización de errores en tiempo real, agilizando así la resolución de problemas críticos en los dispositivos que actúan como clientes. Se han cumplido los objetivos planteados al inicio puesto que la prueba de concepto es funcional y desarrolla el comportamiento esperado.

El sistema se ha puesto en producción en la empresa con el objetivo de testear el sistema en el entorno empresarial no en el académico.

A título personal este trabajo me ha servido para adquirir unos nuevos conocimientos sobre las arquitecturas de los sistemas de Big Data, en especial los sistema de Stream Processing. También he adquirido nuevos conocimientos de administración de sistemas en AWS así como administración del networking.

\section{Trabajo futuro}

El sistema cumple con los objetivos planteados al inicio del proyecto, pero para que esté listo en un entorno empresarial le falta pulir algunos aspectos.
Así pues al sistema se le ha de dotar de monitorización en todos sus módulos, seguridad como mínimo en los módulos que están expuestos a Internet.
Este sistema también ha de servir para recoger eventos de otros dispositivos diferentes a los ubicuos, en este caso se han de adaptar los módulos de recolección y de envío al dispositivo objetivo.