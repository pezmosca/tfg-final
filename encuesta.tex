\chapter{Encuesta conocimientos y competencias en sostenibilidad}
Una vez contestada la encuesta me he dado cuenta de que no he tenido en mente muchos aspectos de sostenibilidad bastante importantes. Por un lado desconocía toda la literatura detrás de la sostenibilidad y los modelos existentes, desconocía el hecho de poder medir cómo de sostenible se es. Mi desconocimiento de la materia en parte puede deberse a que ya llevo años trabajando en la industria y en las empresas que he estado no se ha hecho mucho hincapié en los aspectos de sostenibilidad. Después de haber hecho la encuesta considero que ya empiezo a entender mejor los aspectos que engloba la sostenibilidad y ya empiezo a tenerlos en cuenta en el proyecto, muchos de los aspectos son de sentido común, cosa que puede hacer que se piense que ya se tienen en cuenta cuando se propone un proyecto, pero hay otros aspectos más profundos que es interesante estudiarlos. Por lo que en resumen, mi conocimiento del tema era bastante limitado y me he dado cuenta de que debo de tener en cuenta factores en mi proyecto que pueden ayudar a que este sea sostenible.
Con respecto la calidad de la encuesta, aunque cumple una función de concienciación importante, es mejorable a nivel organizativo. Quizás permitir respuestas más abiertas a las preguntas y organizar las preguntas en subsecciones puede facilitar el hacer resúmenes autoevaluativos posteriormente y darte cuenta qué puntos fuertes y débiles tienes con respecto al tema de la sostenibilidad.
Concluyendo, he de documentarme mejor sobre el tema de la sostenibilidad, no solo porque sea obligación de GEP sino porque es algo importante a nivel social dado que el proyecto tiene un impacto en la sociedad y me he de esforzar de que sea un impacto positivo.