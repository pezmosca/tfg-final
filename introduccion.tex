\chapter{Contextualización del proyecto}

Este proyecto es un Trabajo de Final de Grado de la modalidad C de la especialidad Tecnologías de la Información. El presente trabajo se hace en colaboración con la empresa LudiumLab S.L. la cual es propietaria y creadora del servicio de cloud gaming Play Everywhere. Este servicio consiste en permitir a sus usuarios jugar a juegos diseñados para Windows PC en otros dispositivos como podrían ser smartphones, tabletas o navegadores web. Dado que la empresa da servicio a dispositivos ubicuos, tiene la necesidad de recolectar y procesar alertas para que sus programadores corrijan comportamientos anómalos de la aplicación que se ejecuta en dichos dispositivos. Por lo que este trabajo será un caso de estudio para dar solución a la problemática de la empresa de recolectar y procesar alertas para darle información de valor a sus programadores.

\section{Contexto}
Las aplicaciones ubicuas \cite{Tfg:ubiquitous} se han convertido ya en una constante en la vida diaria. En especial las aplicaciones dirigidas a los smartphones. Cada día aparecen nuevas aplicaciones que hacen que la informática se integre en el entorno de la persona, apareciendo en cualquier lugar y en cualquier momento, este hecho genera nuevos retos para los desarrolladores, como pudiera ser la detección y seguimiento de errores que puedan producirse en tales aplicaciones. Las aplicaciones ubicuas hoy día pueden ejecutarse en un gran número de dispositivos, éste trabajo se centrará en los smartphones y en especial en aquellos que ejecutan Android como sistema operativo.

Para controlar comportamientos anómalos en las aplicaciones, se suelen desarrollar para que, a parte de cumplir su función principal, informen sobre su estado. El registro de actividades de una aplicación es conocido como \textit{log} \cite{Tfg:thelog}. Los sistemas que corren éstas aplicaciones ubicuas también generan logs, en éste trabajo para diferenciar los logs de las aplicaciones, de los logs del sistema, los últimos serán llamados syslogs. Hay logs que directamente informan de un error y de que la aplicación ha dejado de funcionar, para diferenciar a estos logs de los syslogs y los logs, en este trabajo serán llamados crashlogs. El conjunto de logs, syslogs y crashlogs será llamado alertas o eventos.

El hecho de que las aplicaciones ubicuas aparezcan en cualquier lugar y en cualquier momento, hace que muchas veces las aplicaciones ubicuas sean también aplicaciones distribuidas \cite{Tfg:distributedapp}. El controlar la comunicación y la integración de las aplicaciones con otros sistemas puede llegar a producir una cantidad de alertas (logs, syslogs y crashlogs) elevada. Al multiplicar la gran cantidad de alertas producidas por el número de instancias ejecutándose de la aplicación, se obtiene un número más que cuantioso de alertas, las cuales se han de procesar para obtener información sobre el comportamiento de las aplicaciones. Por éste motivo, es cada vez más frecuente entre las empresas el procesar la gran cantidad de alertas producidas por las aplicaciones con soluciones de procesado de Big Data.

Big data \cite{Tfg:bigdata} se refiere al concepto de conjuntos de datos tan masivos que las aplicaciones tradicionales de procesado de datos no son capaces de tratar con ellos. Por lo tanto, las soluciones de procesado de Big Data son todas aquellas capaces de tratar con tal volumen elevado de datos. El proceso que se suele utilizar para tratar Big Data es el de ETL (Extract, Transform and Load) \cite{Tfg:etl}, consiste en extraer los datos desde los sistemas de origen, luego transformarlos de manera oportuna y por último, una vez los datos ya han sido transformados, cargarlos en un sistema destino donde serán consumidos. 

Existen dos técnicas principales para aplicar el ETL en Big Data \cite{Tfg:batchstream}. La primera es el Batch Processing, que consiste en transformar datos que ya han estado almacenados durante un tiempo. La segunda es el Stream Processing, la cual consiste en transformar los datos en tiempo real, es decir, sin que los datos pasen mucho tiempo almacenados antes de ser transformados. Se encuentran dos arquitecturas de Big Data distinguidas que aplican las técnicas mencionadas, una es la arquitectura Lambda \cite{Tfg:lambda}, la cual integra las dos técnicas, y la otra la arquitectura Kappa \cite{Tfg:kappa} en la cual se utiliza mayoritariamente el Stream Processing.


\section{Stakeholders}
Los actores principales de este proyecto son los siguientes:

\begin{enumerate}
	\item \textbf{Director, ponente y desarrollador.} Son los actores que van a hacer que el trabajo cumpla los objetivos propuestos en el tiempo estipulado. El director Juan José Martín Pastor y el ponente Javier Verdú Mulá guiarán el desarrollo del trabajo así como orientar al desarrollador, Antonio Arellano Moral, quien se encargará del diseño y del desarrollo del sistema.
	
	\item \textbf{Audiencia.} El proyecto va dirigido al equipo de desarrolladores y al equipo de operaciones de LudiumLab S.L. Ya que serán ellos quien hagan el uso efectivo del sistema.
	
	\item \textbf{Beneficiarios.} Los beneficiarios del proyecto serán la empresa en general, ya que parte del éxito de la empresa consiste en el buen funcionamiento de sus aplicaciones, y el equipo de desarrolladores y de operaciones en particular, ya que el sistema les facilitará el identificar errores y comportamientos anómalos.
\end{enumerate}
