\chapter{Evaluación}
\section{Métodos de evaluación}

Los métodos de evaluación corresponderían a la fase de verificación del desarrollo en cascada.

La parte de implementación del proyecto se puede dividir en las siguientes fases:

\begin{itemize}
	\item Subsistema de recolección y envío de eventos
	\item Subsistema de recepción e ingesta de eventos
	\item Subsistema de transformación de eventos
	\item Integración con JIRA y ES Stack
\end{itemize}

Se aplica el desarrollo en cascada de forma independiente en cada fase del proyecto por lo que se obtienen cuatro fases de verificación.

La fase de verificación del subsistema de recolección y envío de eventos consiste en la creación de una sencilla aplicación que integre las librerías escogidas en la fase de diseño con el fin de que recolecte y envíe eventos a un servidor desplegado en un entorno de pruebas. Tal servidor consiste en una máquina virtual que simula ser el subsistema de recepción e ingesta de eventos. Se modificará todo aquello de la aplicación que sea erróneo hasta que se consiga el resultado esperado en el servidor. Una vez se consiga el resultado esperado, se podrán exportar las configuraciones a los entornos de producción.

La fase de verificación del subistema de recepción e ingesta de eventos consiste por un lado en configurar en una máquina virtual el módulo de recepción para que sea capaz de recibir enventos y pasarlos al módulo de ingesta. Por otro lado se ha de configurar en otra máquina virtual el módulo de ingesta de eventos para que sea capaz de comunicarse con el módulo de recepción de eventos y almacenar de forma parcial los eventos. Una vez las dos máquinas virtuales se comuniquen entre si, y los mensajes enviados al módulo de recepción se almacenen en el módulo de ingesta, se considerará que la configuración principal es la correcta y se procederá a exportar tal configuración al entorno de producción.

La fase de verificación de verificación del subsistema de transformación de eventos, consiste en configurar en local una máquina virtual con el subsistema de recolección y envío de eventos a la cual se van a ir enviando eventos, tales eventos van a ser consumidos por la aplicación de procesado que hayamos programado y por Logstash, redirigiremos la salida de tales programas a la consola para verificar que la transformación es la correcta, una vez se obtengan los datos esperados, se procederá a desplegar las configuraciones en entornos de producción.

La fase de verificación de la integración con JIRA y ES Stack consiste en montar en local los tres subsistemas, un nodo con JIRA y un nodo con toda la ES Stack, refinar las configuraciones de los módulos hasta que se consiga, recolectar, enviar, recibir, ingerir, transformar y cargar un evento en JIRA y en la ES Stack. En JIRA se tendrá que generar un ticket con información sobre el evento, en ES Stack, Logstash ya ha hecho una transformación del evento, lo pasará a ElasticSearch y Kibana será capaz de consumir el evento apuntando a ElasticSearch. Una vez se consigue el comportamiento deseado, se podrán exportar las configuraciones a los entornos de producción.

Una vez estén todos los módulos en producción también se va a testear que cumpla el comportamiento esperado. Para ello se generarán una serie de eventos, los cuales han de seguir el pipeline hasta llegar a JIRA y Kibana, si JIRA y Kibana muestran lo deseado se puede concluir el testeo.

Además de tales pruebas y para asegurar el correcto avance del diseño del proyecto, se mantendrá un contacto frecuente con el director para que vaya dando su visto bueno. Todas las pruebas y tests las llevará a cabo el desarrollador del proyecto.