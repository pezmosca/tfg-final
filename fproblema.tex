\subsection{Problema}
Para que los desarrolladores puedan ofrecer la mejor experiencia de usuario necesitan recoger información de manera y forma específica sobre el comportamiento de sus aplicaciones, luego procesar esa información para poder identificar posibles comportamientos inesperados de las aplicaciones y recibir avisos con información de valor sobre los comportamientos anómalos encontrados para que puedan ser solucionados.

La empresa encuentra el problema de que esa recolección y procesado de eventos requiere de unos conocimientos sobre como hacer la recolección, el procesado, y que la manera y los medios que se vayan a utilizar se adapten al entorno ya existente en la empresa. Cabe destacar que no existe ningún sistema en la empresa que ya supla la necesidad de recolectar y procesar eventos, por lo que otro problema puede ser la complejidad, ya que el empezar de cero es más complejo que no aprovechar partes existentes. Otro problema se encuentra en el hecho de que la empresa tenderá a crecer y a adquirir nuevas necesidades, por lo que la escalabilidad es otro problema.

\subsection{Objetivo principal}
\begin{enumerate}[A)]
	\item El objetivo principal del trabajo es diseñar, implementar y desplegar un sistema que sea capaz de recolectar eventos de aplicaciones ubicuas, procesarlos e integrarse con herramientas de DevOps.
\end{enumerate}

\subsection{Objetivos específicos}

Para efectuar el objetivo principal, se realizarán los siguientes objetivos específicos:

\begin{enumerate}[a)]
	\item Diseño, implementación y despliegue de un protocolo para la recolección y envío de eventos (logs, syslogs y crashlogs) transparente al usuario.
	
	\item Diseño, implementación y despliegue de un sistema capaz de recibir y almacenar de forma asíncrona un volumen elevado de datos y capaz de integrarse con una herramienta de Big Data Processing.
	
	\item Configuración de una herramienta de Big Data Processing para que se integre en el sistema y almacene los datos transformados donde puedan ser consumidos.
	
	\item Integración de JIRA con el sistema desarrollado para que sea capaz de consumir los datos y crear incidencias dentro de la aplicación gracias a los eventos transformados.
\end{enumerate}

