\section{Cambios en la planificación inicial y en la gestión económica}

Respecto la planificación inicial han habido cambios en la duración del proyecto, ha pasado de durar 540 horas a 420 horas en total. Esta reducción de horas ha sido debida a que la dificultad con la que se había previsto algunas fases del proyecto ha sido menor. Las fases que han reducido sus horas son las siguientes:

\begin{itemize}
	\item \textbf{Subsistema de recolección y envío de eventos} de 90 horas a 66 puesto que la integración de las librerías se ha llevado a cabo sin muchas complicaciones.
	
	\item \textbf{Subsistema de transformación de eventos} de 90 horas a 60 puesto que las herramientas con las que se ha desarrollado dicha transformación no tenían una curva de aprendizaje tan pronunciada como la que se previó.
	
	\item \textbf{Integración con JIRA y ES Stack} de 90 horas a 30 puesto que tanto JIRA como ES están pensados para que su integración sea rápida y fácil.
	
	\item \textbf{Despliegue del sistema} de 18 horas a 6 horas puesto que la mayoría de elementos ya se habían desplegado en las fases anteriores.
	
	\item \textbf{Testeo del sistema en producción} de 18 a 12 horas puesto que gran parte de las pruebas ya se habían realizado en las fases anteriores.
\end{itemize}

La tarea \textit{Integración con JIRA y ES Stack} en el plan inicial tan solo era \textit{Integración con JIRA}, en la planificación final también ha habido una integración con ES Stack.

Estos cambios afectan positivamente a los objetivos y al desarrollo del proyecto puesto que significa que gran parte de los objetivos y el desarrollo ya están hechos.

La reducción de tiempo del proyecto también repercute positivamente en el coste del proyecto, ya que gran parte del coste va ligado a las horas de proyecto. Aunque se cometió un fallo en el cálculo del presupuesto inicial, esta reducción de horas, más la corrección de dicho error ha hecho que el coste total del proyecto disminuya de 33.499,55 € a 24.297,26 €.

En los capítulos \ref{cap:planificacion} y \ref{cap:gestionec} se expone la nueva planificación y los nuevos presupuestos contando esta reducción de horas. En el Anexo \ref{cap:ganttanex} se encuentra el diagrama definitivo en la Figura \ref{fig:gantt}, así como el diagrama que se planteó al inicio en la Figura \ref{fig:ganttini}.
