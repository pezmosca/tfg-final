\chapter{Planificación final}\label{cap:planificacion}
\section{Cambios en la planificación inicial y en la gestión económica}

Respecto la planificación inicial han habido cambios en la duración del proyecto, ha pasado de durar 540 horas a 420 horas en total. Esta reducción de horas ha sido debida a que la dificultad con la que se había previsto algunas fases del proyecto ha sido menor. Las fases que han reducido sus horas son las siguientes:

\begin{itemize}
	\item \textbf{Subsistema de recolección y envío de eventos} de 90 horas a 66 puesto que la integración de las librerías se ha llevado a cabo sin muchas complicaciones.
	
	\item \textbf{Subsistema de transformación de eventos} de 90 horas a 60 puesto que las herramientas con las que se ha desarrollado dicha transformación no tenían una curva de aprendizaje tan pronunciada como la que se previó.
	
	\item \textbf{Integración con JIRA y ES Stack} de 90 horas a 30 puesto que tanto JIRA como ES están pensados para que su integración sea rápida y fácil.
	
	\item \textbf{Despliegue del sistema} de 18 horas a 6 horas puesto que la mayoría de elementos ya se habían desplegado en las fases anteriores.
	
	\item \textbf{Testeo del sistema en producción} de 18 a 12 horas puesto que gran parte de las pruebas ya se habían realizado en las fases anteriores.
\end{itemize}

La tarea \textit{Integración con JIRA y ES Stack} en el plan inicial tan solo era \textit{Integración con JIRA}, en la planificación final también ha habido una integración con ES Stack.

La reducción de tiempo del proyecto también repercute positivamente en el coste del proyecto, ya que gran parte del coste va ligado a las horas de proyecto. Aunque se cometió un fallo en el cálculo del presupuesto inicial, esta reducción de horas, más la corrección de dicho error ha hecho que el coste total del proyecto disminuya de 33.499,55 € a 24.297,26 €.

En los capítulos \ref{cap:planificacion} y \ref{cap:gestionec} se expone la nueva planificación y los nuevos presupuestos contando esta reducción de horas. En el Anexo \ref{cap:ganttanex} se encuentra el diagrama de Gantt definitivo en la Figura \ref{fig:gantt}, así como el diagrama de Gantt que se planteó al inicio en la Figura \ref{fig:ganttini}.


\section{Tabla de tiempo}
En el Tabla \ref{tabla:tiempotareas} se muestran de forma definitiva las tareas a realizar y el tiempo que se va a emplear para realizar tales tareas, en la Sección \ref{cap:gantt} se especificarán con más detalle.

\begin{table}[H]\label{tabla:tiempotareas}
	\centering
	\begin{tabular}{|l|l|l|}
		\hline
		\textbf{Tarea}                               & \textbf{Horas previstas} & \textbf{Horas empleadas}       \\ \hline
		Aprendizaje                                  & 150 horas                & \textbf{150 horas}             \\ \hline
		Subsistema de recolección y envío de eventos & 90 horas                 & \textbf{66 horas}              \\ \hline
		Subsistema de recepción e ingesta de eventos & 90 horas                 & \textbf{90 horas}              \\ \hline
		Subsistema de transformación de eventos      & 90 horas                 & \textbf{60 horas}              \\ \hline
		Integración con JIRA y ES Stack              & 90 horas                 & \textbf{30 horas}              \\ \hline
		Despliegue del sistema                       & 15 horas                 & \textbf{6 horas}               \\ \hline
		Testeo del sistema en producción             & 15 horas                 & \textbf{12 horas}              \\ \hline
		\textbf{Horas totales del proyecto}          & 540 horas                & \underline{\textbf{420 horas}} \\ \hline
	\end{tabular}
	\caption{Tiempo de realización esperado de las tareas}
\end{table}

\section{Restricciones}
Las restricciones no han cambiado con respecto las restricciones del plan inicial que se encuentran en la Sección \ref{sec:restriccionesini} del Capítulo \ref{cap:planini}.

\section{Recursos}
\subsection{Recursos humanos}
Se dispondrá de un desarrollador que es quién lleve a cabo el proyecto. El desarrollador de este proyecto asumirá las tareas de jefe de proyecto, consultor y Site Reliability Engineer (SRE) \cite{Tfg:sre}. El tiempo en horas por semana que dedicará el desarrollador al proyecto serán de 30 horas por semana. Es una medida aproximada que podrá variar por la influencia de diversos factores tales como la carga de trabajo que tenga el desarrollador por causas externas al proyecto.

También se dispone de un director, el cual podrá ser consultado por el desarrollador.

\subsection{Recursos físicos}\label{cap:recfis}
\begin{description}

	\item [Lenovo Y700] Computadora donde se desarrollará el proyecto y se harán las pruebas parciales antes de desplegar el sistema. Se utilizará también para escribir la memoria.
	
	\item [Xiaomi MI A1] Celular con el que se harán las pruebas de recolección y envío de eventos.
	
	\item [GitHub] Servidor remoto de control de versiones. Se utilizará para almacenar las diferentes versiones del software que se desarrolle.
	
	\item [JIRA] Herramienta para administración de tareas de un proyecto. Se publicará documentación de los diversos módulos en ella.
	
	\item [Servidores de producción] En la sección de Gestión Económica se especifican las máquinas necesarias así como el coste de utilizarlas.
\end{description}

\section{Diagrama de Gantt}\label{cap:gantt}
En el Anexo \ref{cap:ganttanex} se encuentra el diagrama de Gantt final Figura \ref{fig:gantt} donde se especifican cada una de las tareas, su fecha de inicio y fin, su duración en horas y el coordinador o responsable de cada tarea. Se ven también las restricciones temporales de cada tarea.

Las tareas marcadas en rojo indican un riesgo alto a que causen posibles desviaciones.
Las tareas marcadas en azul un riesgo medio a que causen posibles desviaciones.

Las tareas marcadas en verde marcan las reuniones de seguimiento con el director, el espacio temporal entre ellas es de quince días de media, estas reuniones empiezan después de la fase de aprendizaje y duran hasta la semana anterior de la entrega del proyecto. En tales reuniones se irán revisando aspectos claves del desarrollo del proyecto así como resolviendo posibles dudas.







