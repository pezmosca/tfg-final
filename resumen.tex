
\renewenvironment{abstract}
{\small
	\begin{center}
		\bfseries \abstractname\vspace{-.5em}\vspace{0pt}
	\end{center}
	\list{}{%
		\setlength{\leftmargin}{5mm}% <---------- CHANGE HERE
		\setlength{\rightmargin}{\leftmargin}%
	}%
	\item\relax}
{\endlist}

\begin{abstract}
	
	Este documento contiene el Trabajo de Final de Grado para el Grado en Ingeniería Informática, especialidad en Tecnologías de la Información por la Facultad de Informática de Barcelona, siendo este trabajo desarrollado en LudiumLab S.L.		
	\\
	Las aplicaciones ubicuas se han convertido ya en una constante en la vida diaria. En especial las aplicaciones dirigidas a los smartphones. Cada día aparecen nuevas aplicaciones que hacen que la informática se integre en el entorno de la persona, apareciendo en cualquier lugar y en cualquier momento. Este hecho genera nuevos retos para los desarrolladores, como pudiera ser la detección y seguimiento de errores que puedan producirse en tales aplicaciones. 
	\\
	Las aplicaciones ubicuas hoy día pueden ejecutarse en un gran número de dispositivos, este trabajo se centrará en los smartphones y en especial en aquellos que ejecutan Android como sistema operativo.
	\\
	Este proyecto tiene como objetivo aportar a la empresa un sistema para la monitorización y la detección de errores en aplicaciones ubicuas, diseñando, implementando y desplegando las diferentes partes de tal sistema.
		
\end{abstract}