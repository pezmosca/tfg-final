\chapter{Alcance}

\section{Meta final}
Dado el tiempo limitado para realizar el Trabajo de Final de Grado, el diseño e implementación del sistema que recopile, reciba, transforme y muestre los datos, se hará lo más rápido y sencillo posible, tan solo para demostrar que la arquitectura es efectiva para el fin propuesto. La meta final se puede dividir en las siguientes metas:

\begin{enumerate}
	
	\item Una sencilla aplicación Android distribuida que genera logs, crashlogs y syslogs, y que usa una librería que de manera transparente al usuario y en el momento más oportuno, recopila eventos y los envía a un sistema para que sean procesados. 
	
	\item Un sistema, accesible por red, de recepción asíncrona y capaz de almacenar un volumen elevado de eventos, que sea escalable, resiliente a fallos en la red, independiente al dispositivo donde se ejecuta la aplicación y que se comunica con el sistema de transformación de eventos.
	
	\item Un sistema de transformación de eventos en tiempo real, siendo el núcleo del sistema una herramienta de Big Data Processing, que transforma en información útil para los desarrolladores, los datos recibidos por el sistema de recepción de eventos. Este sistema deberá almacenar en el lugar más conveniente la información transformada para posteriormente ser consumida por JIRA. La latencia de este sistema será baja en comparación con los sistemas de Batch Processing. Las herramientas que compondrán al sistema serán modernas en la medida de lo posible. La escalabilidad del sistema ha de ser muy elevada ya que luego ha de poder ser utilizado para procesar otro tipo de datos como los derivados del Business Intelligence además unificar de todos los logs que puedan generarse en la empresa en una misma capa.
	
\end{enumerate}

\section{Posibles obstáculos}
\subsection{Desconocimiento de la tecnología}
El desarrollador del proyecto es la primera vez que se enfrenta a tecnologías y arquitecturas de Big Data Processing, por lo que el avance del proyecto depende del conocimiento que adquiere de forma autodidacta. Una confusión conceptual podría retrasar el avance del proyecto. 

\textbf{Posible solución:} Reuniones frecuentes con el director para en el caso que el desarrollador desconociera algún concepto, el director le pueda guiar sobre la bibliografía a revisar.

\subsection{Errores de configuración}
El proyecto lo componen diferentes sistemas que están conectados entre sí por red. Una configuración errónea de alguno de ellos puede hacer que el avance del proyecto se retrase.

\textbf{Posible solución:} Comprobar de forma independiente la configuración de cada sistema antes de empezar a configurar el siguiente sistema. La arquitectura del sistema a implementar hace posible configurar un sistema y poder probar dicha configuración antes de continuar configurando otro sistema. De esta manera, se puede localizar antes cualquier error en la configuración y comentarlo con el director si es necesario.

\subsection{Errores en el código}
Algo común cuando se desarrolla software son los errores en él. Un error en el código de la recopilación de datos, el envío, la transformación o su muestreo puede hacer que el avance del proyecto se vea afectado.

\textbf{Posible solución:} El uso de sistemas de control de versiones de software ayuda a encontrar de forma rápida errores en el código. Probar el software con baterías de tests a medida que se va desarrollando puede ayudar también a localizar rápidamente los errores.

\subsection{Integración con JIRA}
La integración con JIRA puede ser un problema ya que JIRA no está pensado para integrarse con nuestro sistema.

\textbf{Posible solución:} Es sabido que empresas han conseguido integrar JIRA con sistemas similares al de este proyecto[17], por lo que si la integración fuese un problema, indagar un poco más en la documentación o incluso contactar con la comunidad de desarrolladores para que guíen en los pasos a seguir.

\subsection{Restricciones temporales}
Dado que el Trabajo de Fin de Grado tiene unas fechas límites, un posible retraso podría ser un problema.

\textbf{Posible solución:} Planificar las tareas a llevar a cabo con cierto margen y contando que pueden surgir contratiempos.

\subsection{Actualización de los servicios utilizados}
Al utilizar herramientas actuales y en desarrollo es posible que aparezca alguna actualización que corrija algún aspecto fundamental de dicha herramienta y cambie la manera en que se relaciona con otros elementos del sistema. Se considera un obstáculo ya que puede hacer repetir tareas ya acabadas.

\textbf{Posible solución:} Escoger las herramientas, que componen el sistema, mejor valoradas por la comunidad y en versiones estables.




