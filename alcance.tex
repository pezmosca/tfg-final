\subsection{Resultados esperados}
Los resultados que se esperan obtener con este Trabajo de Final de Grado son los siguientes:

\begin{enumerate}
		
	\item Integración de una o más librerías que sean capaces de recolectar logs y crashlogs en un dispositivo ubicuo y enviarlos a un sistema externo. Esta recolección y envío se ha de realizar de forma transparente al usuario en el momento que se considere oportuno. 
	
	\item Un sistema, accesible por red, de recepción asíncrona y capaz de almacenar un volumen elevado de eventos, que sea escalable, resiliente a fallos en la red, independiente al dispositivo donde se ejecuta la aplicación y que se comunica con el sistema de transformación de eventos.
	
	\item Un sistema de transformación de eventos en tiempo real, siendo el núcleo del sistema una herramienta de Big Data Processing, que transforma en información útil para los desarrolladores, los datos recibidos por el sistema de recepción de eventos. Este sistema deberá almacenar en el lugar más conveniente la información transformada ya sea en JIRA o en Elastic Search. La latencia de este sistema será baja en comparación con los sistemas de Batch Processing. Las herramientas que compondrán el sistema serán modernas en la medida de lo posible. La escalabilidad del sistema ha de ser muy elevada, ya que luego ha de poder ser utilizado para procesar otro tipo de datos como los derivados del Business Intelligence, además unificar de todos los logs que puedan generarse en la empresa en una misma capa.
	
\end{enumerate}





