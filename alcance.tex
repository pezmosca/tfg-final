\subsection{Meta final}
La meta final se puede dividir en las siguientes metas:

\begin{enumerate}
	
	\item Una sencilla aplicación Android distribuida que genera logs, crashlogs y syslogs, y que usa una librería que de manera transparente al usuario y en el momento más oportuno, recopila eventos y los envía a un sistema para que sean procesados. 
	
	\item Un sistema, accesible por red, de recepción asíncrona y capaz de almacenar un volumen elevado de eventos, que sea escalable, resiliente a fallos en la red, independiente al dispositivo donde se ejecuta la aplicación y que se comunica con el sistema de transformación de eventos.
	
	\item Un sistema de transformación de eventos en tiempo real, siendo el núcleo del sistema una herramienta de Big Data Processing, que transforma en información útil para los desarrolladores, los datos recibidos por el sistema de recepción de eventos. Este sistema deberá almacenar en el lugar más conveniente la información transformada ya sea en JIRA o en Elastic Search. La latencia de este sistema será baja en comparación con los sistemas de Batch Processing. Las herramientas que compondrán al sistema serán modernas en la medida de lo posible. La escalabilidad del sistema ha de ser muy elevada ya que luego ha de poder ser utilizado para procesar otro tipo de datos como los derivados del Business Intelligence además unificar de todos los logs que puedan generarse en la empresa en una misma capa.
	
\end{enumerate}





