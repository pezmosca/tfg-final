\chapter{Sostenibilidad}

\section{Matriz de sostenibilidad}
La Tabla \ref{tab:matsos} es la llamada matriz de sostenibilidad, esta resume las secciones \ref{sec:imec}, \ref{sec:imam} y \ref{sec:imsoc}. Los criterios seguidos para determinar las puntuaciones son: cómo se ha llevado a cabo el proyecto, la utilidad del proyecto y los riesgos que puede suponer tal proyecto.
El objetivo intrínseco del proyecto es conseguir un mejor rendimiento de los productos de la empresa para aumentar su beneficio económico, por lo que en la dimensión económica obtiene un 10/10 ya que es viable y se pueden apreciar ciertas mejoras con respecto la competencia. En la dimensión ambiental se ha intentado buscar soluciones que no afecten directamente sobre el medio-ambiente de forma negativa, pero al delegar a otras compañías ciertos servicios del proyecto no se puede estar completamente seguro de cumplan su compromiso ambiental, por lo que la puntuación es 6/10. En la dimensión social existente varios riesgos de impactar negativamente, pero el desarrollador del proyecto se compromete a ser transparente en este aspecto, cosa que puede ser diferencial con la competencia por lo que la puntuación que se obtiene es de 8/10.

\begin{table}[H]\label{tab:matsos}
	\centering
	\begin{tabular}{|c|c|c|}
		\hline
		\textbf{Dimensión económica} & \textbf{Dimensión ambiental} & \textbf{Dimensión social}  \\ \hline
		10/10                 &        6/10                  &       8/10                 \\ \hline
		\multicolumn{3}{|c|}{\textbf{Total 23/30}} \\ \hline
	\end{tabular} 
	\caption{Matriz de sostenibilidad}
\end{table}

\section{Impacto económico}\label{sec:imec}
 
Los costes para este proyecto son más bajos que los de la competencia ya que las necesidades y los recursos también son más bajos. El hecho de que se plantee el problema de forma diferente que las demás compañías, puede hacer que la necesidad de recursos sea menor y este hecho puede ser el que sea mejor a las soluciones existentes a nivel económico, ya que se está diseñando un sistema desde cero en vez de adaptar una solución concreta para que resuelva un problema similar al que ya soluciona. Como resultado se es más eficiente a la hora de adquirir servidores y servicios.

\section{Impacto ambiental}\label{sec:imam}
El indicador de impacto ambiental de este proyecto es principalmente el consumo eléctrico. En la sección \ref{sec:gastosin} se ha detallado el consumo eléctrico de la realización del proyecto, en el momento en que el proyecto ha pasado a producción, el consumo eléctrico ya no depende íntegramente de la empresa. Se planteó el hecho de comprar servidores y colocarlos en la empresa en vez de alquilar por horas los servidores, esta idea se desestimó y una de las razones fue el consumo eléctrico. AWS es una empresa especializada en el alquiler de servidores y como tal busca su máximo beneficio, por lo que AWS intentará disminuir al máximo su consumo eléctrico dado que reduciendo el consumo aumentan sus beneficios. Por lo tanto, con AWS se está seguro de que se tiene el mínimo consumo eléctrico y por ende el mínimo impacto ambiental.
El proyecto se ha diseñado en vistas a que pueda ser utilizado para procesar otro tipo de datos diferente a los eventos, como pudiera ser los derivados del BI, por lo que ya en el planteamiento inicial del proyecto se pensó en reutilizar recursos a posteriori, a priori no se han podido reutilizar recursos dado que no existían. 
Actualmente las empresas que se disponen a procesar eventos tienden a unificar su capa de eventos con su capa de BI, algo muy parecido que lo que intenta este proyecto. La peculiaridad de este proyecto es el hecho de que primero se diseña un sistema para procesar eventos que luego tendrá que soportar procesar datos de BI, cuando las otras empresas plantearon, por diversas razones, el problema isómero, es decir, diseñaron un sistema de procesado de BI que luego están adaptando para procesar eventos. El hecho de no adaptar ningún sistema, sino diseñar un sistema que admita el procesado de los dos tipos de datos puede eliminar ciertos sobrecostes, como máquinas extras resultado de la adaptación y no de un diseño inteligente. Por lo que esta solución puede ser mejor ambientalmente que las existentes, dado que se pueden reducir el uso de máquinas extras y como consecuencia un menor gasto eléctrico.

\section{Impacto social}\label{sec:imsoc}
El impacto personal del proyecto ha sido elevado, ya que documentándome para entender el estado del arte, me he dado cuenta de la cantidad de recursos que se invierten para proyectos parecidos para el mío y el uso que se le puede dar desde el punto de vista del dueño de los datos. Mi proyecto se centra en la recolección de eventos para mejorar la eficiencia del trabajo de los desarrolladores, pero también el sistema tendrá que ser compatible para procesar datos de BI.
Los datos de BI al fin y al cabo lo que están midiendo es el comportamiento del usuario con respecto nuestro producto. Esto me genera ciertos dilemas éticos con respecto el trabajo futuro y cómo ha de evolucionar el proyecto. Es cierto que las empresas del sector suelen avisar de que recolectan datos del uso que se hace de sus productos, pero desde mi punto de vista los usuarios no son conscientes de qué datos se están recolectando ni cómo se están utilizando, las empresas del sector no acaban de asegurar que hacen con tales datos una vez han sido procesados.
Creo que tengo el deber ético de ser lo más transparente posible con el uso que haré de esos datos ya que con la información que se recolecta se está muy cerca de vulnerar una intimidad que los usuarios no saben que tienen. Por lo que algo que puede mejorar socialmente mi solución es la transparencia con respecto a los datos recolectados. Una vez acabado el Trabajo de Fin de Grado, este proyecto continuará por lo que mi deber debería ser también una labor de concienciación una vez el Trabajo de Fin de Grado haya terminado.
Aunque el sistema planteado tiene sus peligros a nivel de privacidad, existe la necesidad de que exista, ya que analizando los eventos podemos mejorar las aplicaciones y corregir, por ejemplo, fallos de seguridad que podrían afectar más gravemente a la privacidad que el propio sistema. Este sistema nos proporciona un gran poder, por lo que conlleva una gran responsabilidad de que su uso sea el correcto y más beneficioso para la sociedad en general.


