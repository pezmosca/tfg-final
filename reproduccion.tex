\chapter{Reproducción del sistema}

En el siguiente Anexo se detalla como reproducir los módulos de recepción, ingesta, transformación y almacenamiento en un solo nodo. Algunas configuraciones difieren de las del sistema en producción por el hecho de reproducirse en solo un nodo. Se supone un nodo con Ubuntu 16.04 Server Edition y con JRE 1.8.0 instalado.

Los módulos de recepción e ingesta están compuestos por librerías de Android, el siguiente enlace es un repositorio de una aplicación Android que se ha utilizado para testear que los módulos se integran correctamente:

\href{https://github.com/pezmosca/LogBackToKafka.git}{https://github.com/pezmosca/LogBackToKafka.git}

\section{Módulo de recepción}
\subsection{Adquisición de la herramientas}
Para adquirir la herramienta de este módulo se ha de rellenar un formulario en la siguiente web:

\href{https://www.confluent.io/download/}{https://www.confluent.io/download}

Se descargará en el home de Ubuntu para simplificar la reproducción.

\subsection{Instalación de las herramientas}
La herramienta será instalada en el home de Ubuntu para simplificar la reproducción. Suponiendo que la herramienta está en el home y suponiendo la versión 4.1.1:

\begin{lstlisting}[language=Bash]
tar xzf confluent-oss-4.1.1-2.11.tar.gz
mv confluent-oss-4.1.1-2.11.tar.gz confluent

sudo systemctl enable kafka-rest.service
sudo systemctl start kafka-rest.service
\end{lstlisting}

\subsubsection{kafka-rest.service}
\begin{lstlisting}[language=Bash, breaklines=true]
[Unit]
Description=Apache Kafka HTTP Proxy
Documentation=https://docs.confluent.io/current/kafka-rest/docs/index.html
Requires=network.target remote-fs.target
After=network.target remote-fs.target

[Service]
Type=simple
User=user
Group=user
ExecStart=/home/user/confluent/bin/kafka-rest-start /home/user/confluent/etc/kafka-rest/kafka-rest.properties
ExecStop=/home/user/confluent/bin/kafka-rest-stop

[Install]
WantedBy=multi-user.target
\end{lstlisting}

\subsection{Configuración de las herramientas}
La configuración por defecto es valida para este ejemplo.

%%%%%%%%%%%%%%%%%%%%%%%%

\section{Módulo de ingesta}
\subsection{Adquisición de las herramientas}
Se descargará en el home de Ubuntu para simplificar la reproducción.
\begin{lstlisting}[language=Bash]
wget http://apache.rediris.es/kafka/1.1.0/kafka_2.11-1.1.0.tgz
\end{lstlisting}

\subsection{Instalación de las herramientas}
La herramienta será instalada en el home para simplificar la reproducción. Suponiendo que la herramienta está en el home y suponiendo la versión 1.1.0:
\begin{lstlisting}[language=Bash]
tar xzf kafka_2.11-1.1.0.tgz
mv kafka_2.11-1.1.0.tgz kafka

sudo systemctl enable zookeeper.service
sudo systemctl start zookeeper.service

sudo systemctl enable kafka.service
sudo systemctl start kafka.service
\end{lstlisting}

\subsubsection{zookeeper.service}
\begin{lstlisting}[language=Bash, breaklines=true]
[Unit]
Description=Apache Zookeeper server (Kafka)
Documentation=http://zookeeper.apache.org
Requires=network.target remote-fs.target 
After=network.target remote-fs.target

[Service]
Type=simple
User=user
Group=user
ExecStart=/home/user/kafka/bin/zookeeper-server-start.sh /home/user/kafka/config/zookeeper.properties
ExecStop=/home/user/kafka/bin/zookeeper-server-stop.sh

[Install]
WantedBy=multi-user.target
\end{lstlisting}

\subsubsection{kafka.service}
\begin{lstlisting}[language=Bash, breaklines=true]
[Unit]
Description=Apache Kafka server (broker)
Documentation=http://kafka.apache.org/documentation.html
Requires=network.target remote-fs.target
After=network.target remote-fs.target

[Service]
Type=simple
User=user
Group=user
ExecStart=/home/user/kafka/bin/kafka-server-start.sh /home/user/kafka/config/server.properties
ExecStop=/home/user/kafka/bin/kafka-server-stop.sh

[Install]
WantedBy=multi-user.target
\end{lstlisting}

\subsection{Configuración de las herramientas}
La configuración por defecto es valida para este ejemplo.

%%%%%%%%%%%%%%%%%%%%%%%%

\section{Módulo de procesado}
\subsection{Adquisición de las herramientas}
Se descargará en el home de Ubuntu para simplificar la reproducción. De Kafka Streams se descargarán los jars utilizados para testear la solución, hay diversos jars para poder testear por separado cada una de las acciones que hace la aplicación con Kafka Streams, pero se podría implementar con threads en un solo jar.
\begin{lstlisting}[language=Bash, breaklines=true]
wget https://github.com/pezmosca/empty/raw/master/jars.tar.gz
wget https://artifacts.elastic.co/downloads/logstash/logstash-6.3.0.tar.gz
\end{lstlisting}

\subsection{Instalación de las herramientas}
La herramienta será instalada en el home para simplificar la reproducción. Suponiendo que la herramienta está en el home y suponiendo la versión 6.3.0:
\begin{lstlisting}[language=Bash, breaklines=true]
tar xzf logstash-6.3.0.tar.gz
mv logstash-6.3.0.tar.gz logstash
tar xzf jars.tar.gz

sudo systemctl enable logstash.service
sudo systemctl start logstash.service

sudo systemctl enable kafkastreams.service
sudo systemctl start kafkastreams.service
\end{lstlisting}
\subsubsection{logstash.service}
\begin{lstlisting}[language=Bash, breaklines=true]
[Unit]
Documentation=https://www.elastic.co/products/logstash
After=network.target
ConditionPathExists=/home/ec2-user/logstash/tfg-pipeline.conf

[Service]
User=user
ExecStart=/home/user/logstash/bin/logstash -f /home/user/logstash/tfg-pipeline.conf --config.reload.automatic

[Install]
WantedBy=multi-user.target
\end{lstlisting}

\subsubsection{kafkastreams.service}
\begin{lstlisting}[language=Bash]
[Unit]
Description=Kafka Streams Service
Requires=network.target remote-fs.target
After=network.target remote-fs.target

[Service]
User=user
Group=user
ExecStart=/home/user/jars/start.sh

[Install]
WantedBy=multi-user.target
\end{lstlisting}

\subsection{Configuración de las herramientas}
La configuración por defecto no es valida para este ejemplo, se ha de generar una pipeline para Logstash.


\subsubsection{Logstash}
Se ha de generar una pipeline para procesar los eventos. Tal pipeline se ha de crear en \textasciitilde/logstash/tfg-pipeline.conf y el contenido ha de ser el siguiente:

\begin{lstlisting}
input {
  kafka {
    bootstrap_servers => "localhost:9092"
    topics => ["logs", "crashout"]
    group_id => "tfg-logstash"
    codec => json
    decorate_events => true
  }
}

filter {
  if [@metadata][kafka][topic] == "logs" {
    mutate {
      rename => { "T" => "ExecutionTime"
      "T2" => "Timestamp"
      "L" => "Level"
      "TN" => "ThreadName"
      "LN" => "Class"
      "M" => "Message"
      }
    }
  }
}

output {

  if[@metadata][kafka][topic] == "logs" {
    elasticsearch {
      hosts => [ "localhost:9200" ]
      index => "logs"
    }
  }


  if[@metadata][kafka][topic] == "crashout" {
    elasticsearch {
      hosts => [ "localhost:9200" ]
      index => "crashes"
    }
  }
}
\end{lstlisting}

%%%%%%%%%%%%%%%%%%%%%%%%

\section{Módulo de almacenamiento}
\subsection{Adquisición de las herramientas}
Las herramientas para este módulo se descargarán en el home de Ubuntu para simplificar la reproducción.
\begin{lstlisting}[language=Bash, breaklines=true]
wget https://artifacts.elastic.co/downloads/elasticsearch/elasticsearch-6.3.0.tar.gz
wget https://artifacts.elastic.co/downloads/kibana/kibana-6.3.0-linux-x86_64.tar.gz
\end{lstlisting}

\subsection{Instalación de las herramientas}
Las herramientas serán instalada en el home para simplificar la reproducción, suponiendo que las herramientas están en el home y suponiendo la versión 6.3.0 de Elastic Search, Logstash y Kibana, y la 7.10.1 de JIRA. Para adquirir JIRA se ha de acceder mediante un navegador web a \href{https://es.atlassian.com/software/jira/download}{https://es.atlassian.com/software/jira/download} y descargarlo.
\begin{lstlisting}[language=Bash]
tar xzf elasticsearch-6.3.0.tar.gz
mv elasticsearch-6.3.0.tar.gz elasticsearch

tar xzf kibana-6.3.0.tar.gz
mv kibana-6.3.0.tar.gz kibana

sudo systemctl enable elasticsearch.service
sudo systemctl enable kibana.service

sudo chmod +x atlassian-jira-software-7.10.1-x64.bin
sudo ./atlassian-jira-software-7.10.1-x64.bin
\end{lstlisting}

\subsubsection{elasticsearch.service}\label{cap:elasticsearch}
\begin{lstlisting}[language=Bash]
[Unit]
Documentation=https://www.elastic.co/products/elasticsearch
After=network.target

[Service]
User=user
Group=user
LimitMEMLOCK=infinity
LimitNOFILE=65536
ExecStart=/home/user/elasticsearch/bin/elasticsearch

[Install]
WantedBy=multi-user.target
\end{lstlisting}

\subsubsection{kibana.service}\label{cap:kibana}
\begin{lstlisting}[language=Bash]
[Unit]
Description=Kibana
After=network.target

[Service]
ExecStart=/home/user/kibana/bin/kibana
Type=simple
Restart=always

[Install]
WantedBy=default.target
\end{lstlisting}


\subsection{Configuración de las herramientas}
La configuración por defecto es valida para este ejemplo.






